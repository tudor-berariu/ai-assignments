\documentclass[12pt]{article}

\usepackage{xcolor}
\usepackage[romanian]{babel}
\usepackage{graphicx}
\usepackage{import}
\usepackage[utf8x]{inputenc}
\usepackage{multicol}

\title{Inferențe în Rețele Bayesiene}
\author{Tudor Berariu \\ \emph{tudor.berariu@gmail.com}}

\begin{document}

\maketitle

\subsection*{1. Mai multe cauze ale aceluiași eveniment}
\label{mc}

\subsubsection*{Trei cauze}
\label{sec:trei}

Fie rețeaua bayesiană din Figura~\ref{fig:bayes1}. Aceasta este
descrisă de următorii parametri:
\begin{multicol}{2}
  \begin{eqnarray*}
    & & P(A_1) \\
    & & P(A_2) \\
    & & P(A_3) \\
    & & P(B \vert A_1, A_2, A_3) \\
    & & P(B \vert A_1, A_2, \neg A_3) \\
    & & P(B \vert A_1, \neg A_2, A_3) \\
    & & \ldots \\
    & & P(B \vert \neg A_1, \neg A_2, \neg A_3)
  \end{eqnarray*}
  \begin{figure}[h]
    \centering
    \def\svgwidth{8cm}
    \import{graphics/}{bayes1.pdf_tex}
    \caption{Rețea Bayesiană}
    \label{fig:bayes1}
  \end{figure}
\end{multicol}



\subsubsection*{$N$ cauze}
\label{sec:n}



\end{document}
