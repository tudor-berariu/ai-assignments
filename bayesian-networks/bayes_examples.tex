\documentclass[12pt]{article}

\usepackage{xcolor}
\usepackage[romanian]{babel}
\usepackage{graphicx}
\usepackage{import}
\usepackage[utf8x]{inputenc}
\usepackage{multicol}
\usepackage{amsmath}
%%\usepackage{showlabels}

\title{Inferențe în Rețele Bayesiene}
\author{Tudor Berariu \\ \emph{tudor.berariu@gmail.com}}

\begin{document}

\maketitle

\subsection*{1. Mai multe cauze ale aceluiași eveniment}
\label{mc}

\subsubsection*{Trei cauze}
\label{sec:trei}

Fie rețeaua bayesiană din figura de mai jos. Aceasta este
descrisă de următorii parametri:
\begin{multicols}{2}
  \begin{eqnarray*}
    & & P(A_1) \\
    & & P(A_2) \\
    & & P(A_3) \\
    & & P(B \vert A_1, A_2, A_3) \\
    & & P(B \vert A_1, A_2, \neg A_3) \\
    & & P(B \vert A_1, \neg A_2, A_3) \\
    & & \ldots \\
    & & P(B \vert \neg A_1, \neg A_2, \neg A_3)
  \end{eqnarray*}
  \vfill
  \def\svgwidth{6cm}
  \import{graphics/}{bayes1.pdf_tex}
\end{multicols}

\paragraph{Problema}

Să prespupunem că dorim să calculăm $P(A_1 \vert B)$.

Conform Teoremei lui Bayes:
\begin{equation}
  \label{eq:1}
  P(A_1 \vert B) = \dfrac{P(B \vert A_1) \cdot P(A_1)}{P(B)}
\end{equation}
În Ecuația~\ref{eq:1} avem două cantități necunoscute: $P(B \vert
A_1)$ și $P(B)$.

Aplicând formula probailităților totale:
\begin{equation}
  \label{eq:2}
  P(B \vert A_1) = P(B \vert A_1, A_2) \cdot P(A_2 \vert A_1) + P(B \vert A_1, \neg A_2) \cdot P(\neg A_2 \vert A_1)
\end{equation}

Dar, cum [fără a fi $B$ observat] $A_1$ și $A_2$ sunt condițional
independente, $P(A_2 \vert A_1) = P(A_2)$ și $P(\neg A_2 \vert A_1) =
P(\neg A_2)$, iar Ecuația\ref{eq:2} devine:
\begin{equation}
  \label{eq:3}
  P(B \vert A_1) = P(B \vert A_1, A_2) \cdot P(A_2) + P(B \vert A_1, \neg A_2) \cdot P(\neg A_2)
\end{equation}

Aplicând din nou formula probabilităților totale pentru a factoriza după $A_3$:

\begin{eqnarray*}
  \label{eq:4}
  P(B \vert A_1) & = & P(B \vert A_1, A_2) \cdot P(A_2) + P(B \vert A_1, \neg A_2) \cdot P(\neg A_2) \\
  P(B \vert A_1) & = & \big( P(B \vert A_1, A_2, A_3) \cdot P(A_3 \vert A_1, A_2) + \\
  & & 
  \quad P(B \vert A_1, A_2, \neg A_3) \cdot P(\neg A_3 \vert A_1, A_2) \big) \cdot P(A_2) + \\
  &  & \big( P(B \vert A_1, \neg A_2, A_3) \cdot P(A_3 \vert A_1, \neg A_2) + \\
  & & \quad P(B \vert A_1, \neg A_2, \neg A_3) \cdot P(\neg A_3 \vert A_1, \neg A_2) \big) \cdot P(\neg A_2)
\end{eqnarray*}

\subsubsection*{$N$ cauze}
\label{sec:n}



\end{document}
