\documentclass[12pt]{article}

\usepackage[usenames,dvipsnames]{xcolor}
\usepackage[romanian]{babel}
\usepackage{graphicx}
\usepackage{import}
\usepackage[utf8x]{inputenc}
\usepackage{multicol}
\usepackage{amsmath}
\usepackage{framed}
%%\usepackage{showlabels}

\title{Inferențe în Rețele Bayesiene}
\author{Tudor Berariu \\ \emph{tudor.berariu@gmail.com}}

\begin{document}

\maketitle

\section{Mai multe cauze ale aceluiași eveniment}
\label{mc}

\subsection{Trei cauze}
\label{sec:trei}

\subsubsection*{Problema}
\label{prob1}

Fie rețeaua bayesiană din figura de mai jos. Aceasta este
descrisă de următorii [$1 + 1 + 1 + 2^3 = 11$] parametri:
\begin{multicols}{2}
  \begin{eqnarray*}
    & & P(A_1) \\
    & & P(A_2) \\
    & & P(A_3) \\
    & & P(B \vert A_1, A_2, A_3) \\
    & & P(B \vert A_1, A_2, \neg A_3) \\
    & & P(B \vert A_1, \neg A_2, A_3) \\
    & & \ldots \\
    & & P(B \vert \neg A_1, \neg A_2, \neg A_3)
  \end{eqnarray*}
  \vfill
  \def\svgwidth{6cm}
  \import{graphics/}{bayes1.pdf_tex}
\end{multicols}

Să prespupunem că dorim să calculăm $P(A_1 \vert B)$.

\subsubsection*{Primul pas}
\label{step1}

Conform Teoremei lui Bayes:
\begin{equation}
  \label{eq:1}
  P(A_1 \vert B) = \dfrac{{\color{OliveGreen} P(B \vert A_1)} \cdot P(A_1)}{{\color{DarkOrchid}P(B)}}
\end{equation}
În Ecuația~\ref{eq:1} avem două cantități necunoscute: ${\color{OliveGreen} P(B \vert A_1)}$ și ${\color{DarkOrchid}P(B)}$.

\subsubsection*{Calculul lui ${\color{OliveGreen}P(B \vert A_1)}$}
\label{step2}

Putem introduce o variabilă nouă, aplicând formula probailităților
totale.

\begin{framed}
  \begin{equation*}
    P(X) = \displaystyle\sum_{y}P(X \vert Y=y) \cdot P(Y=y)
  \end{equation*}
  Pentru evenimente binare:
  \begin{equation*}
    P(X) = P(X \vert Y) \cdot P(Y) + P(X \vert \neg Y) \cdot P(\neg Y)
  \end{equation*}
\end{framed}


 O introducem pe $A_2$:
\begin{equation}
  \label{eq:2}
  {\color{OliveGreen} P(B \vert A_1)} = P(B \vert A_1, A_2) \cdot P(A_2 \vert A_1) + P(B \vert A_1, \neg A_2) \cdot P(\neg A_2 \vert A_1)
\end{equation}

Dar, cum [fără a fi $B$ observat] $A_1$ și $A_2$ sunt condițional
independente, $P(A_2 \vert A_1) = P(A_2)$ și $P(\neg A_2 \vert A_1) =
P(\neg A_2)$, iar Ecuația~\ref{eq:2} devine:
\begin{equation}
  \label{eq:3}
  {\color{OliveGreen} P(B \vert A_1)} = P(B \vert A_1, A_2) \cdot P(A_2) + P(B \vert A_1, \neg A_2) \cdot P(\neg A_2)
\end{equation}

Aplicând din nou formula probabilităților totale pentru a factoriza
după $A_3$:

\begin{eqnarray*}
  \label{eq:4}
  {\color{OliveGreen} P(B \vert A_1)} & = & {\color{red} P(B \vert A_1, A_2)} \cdot P(A_2) + {\color{blue} P(B \vert A_1, \neg A_2)} \cdot P(\neg A_2) \\
  {\color{OliveGreen} P(B \vert A_1)} & = & \big( {\color{red} P(B \vert A_1, A_2, A_3) \cdot P(A_3 \vert A_1, A_2) + }\\
  & & 
  {\color{red} \quad P(B \vert A_1, A_2, \neg A_3) \cdot P(\neg A_3 \vert A_1, A_2)} \big) \cdot P(A_2) + \\
  &  & \big( {\color{blue} P(B \vert A_1, \neg A_2, A_3) \cdot P(A_3 \vert A_1, \neg A_2) + }\\
  & & {\color{blue}\quad P(B \vert A_1, \neg A_2, \neg A_3) \cdot P(\neg A_3 \vert A_1, \neg A_2)} \big) \cdot P(\neg A_2)
\end{eqnarray*}

Dar, aplicând același raționament ca înainte, $A_3$ este condițional
independent de $A_1$ și $A_2$ atunci când $B$ nu este observat și
atunci:
\begin{itemize}
\item $P(A_3 \vert A_1, A_2) = P(A_3)$
\item $P(\neg A_3 \vert A_1, A_2) = P(\neg A_3)$
\item $P(A_3 \vert A_1, \neg A_2) = P(A_3)$
\item $P(\neg A_3 \vert A_1, \neg A_2) = P(\neg A_3)$
\end{itemize}

și atunci rescriem ecuația anterioară:

\begin{eqnarray*}
  \label{eq:5}
  {\color{OliveGreen} P(B \vert A_1)} & = & \big( {\color{red} P(B \vert A_1, A_2, A_3) \cdot P(A_3) + }\\
  & & 
  {\color{red} \quad P(B \vert A_1, A_2, \neg A_3) \cdot P(\neg A_3)} \big) \cdot P(A_2) + \\
  &  & \big( {\color{blue} P(B \vert A_1, \neg A_2, A_3) \cdot P(A_3) + }\\
  & & {\color{blue}\quad P(B \vert A_1, \neg A_2, \neg A_3) \cdot P(\neg A_3)} \big) \cdot P(\neg A_2) \\
  & = & P(B \vert A_1, A_2, A_3) \cdot P(A_2) \cdot P(A_3) + \\
  &  & P(B \vert A_1, A_2, \neg A_3) \cdot P(A_2) \cdot P(\neg A_3) + \\
  &  & P(B \vert A_1, \neg A_2, A_3) \cdot P(\neg A_2) \cdot P(A_3) + \\
  &  & P(B \vert A_1, \neg A_2, \neg A_3) \cdot P(\neg A_2) \cdot P(\neg A_3)
\end{eqnarray*}

Înlocuind ultimul rezultat în Ecuația~\ref{eq:2}:

\begin{eqnarray*}
  P(A_1 | B)   & = & \Bigg(P(B \vert A_1, A_2, A_3) \cdot P(A_2) \cdot P(A_3) + \\
  &  & P(B \vert A_1, A_2, \neg A_3) \cdot P(A_2) \cdot P(\neg A_3) + \\
  &  & P(B \vert A_1, \neg A_2, A_3) \cdot P(\neg A_2) \cdot P(A_3) + \\
  &  & P(B \vert A_1, \neg A_2, \neg A_3) \cdot P(\neg A_2) \cdot P(\neg A_3)\Bigg) \cdot \dfrac{P(A_1)}{{\color{DarkOrchid}P(B)}}
\end{eqnarray*}


\subsubsection*{${\color{DarkOrchid}P(B)}$ este necunoscut}
\label{sec:step3}

Mai rămâne o problemă: calculul lui
${\color{DarkOrchid}P(B)}$. Folosind aceeași strategie de mai sus
intuim că ${\color{DarkOrchid}P(B)}$ s-ar exprima printr-o sumă de 8
termeni de tipul $P(B \vert A_1, A_2, A_3) \cdot P(A_1) \cdot P(A_2)
\cdot P(A_3)$. Mai există o soluție prin care putem evita această
desfășurare explicită a lui ${\color{DarkOrchid}P(B)}$ în funcție de
cauzele lui (evenimentele ce condiționează pe) $B$: calculăm $P(\neg
A_1 | B)$.

Urmând aceiași pași ca mai devreme, calculăm:

\begin{eqnarray*}
  P(\neg A_1 | B)   & = & \Bigg(P(B \vert \neg A_1, A_2, A_3) \cdot P(A_2) \cdot P(A_3) + \\
  &  & P(B \vert \neg A_1, A_2, \neg A_3) \cdot P(A_2) \cdot P(\neg A_3) + \\
  &  & P(B \vert \neg A_1, \neg A_2, A_3) \cdot P(\neg A_2) \cdot P(A_3) + \\
  &  & P(B \vert \neg A_1, \neg A_2, \neg A_3) \cdot P(\neg A_2) \cdot P(\neg A_3)\Bigg) \cdot \dfrac{P(\neg A_1)}{{\color{DarkOrchid}P(B)}}
\end{eqnarray*}

Știind că $P(A_1 \vert B) + P(\neg A_1 \vert B) = 1$, obținem un
sistem de ecuații pe care îl putem rezolva fără a mai calcula
${\color{DarkOrchid}P(B)}$.

\subsection{$N$ cauze}
\label{sec:n_causes}

Dacă am avea nu 3, ci $N$ evenimente $A_i$ ce condiționează
evenimentul $B$:

\begin{eqnarray*}
  P(A_i \vert B) & = &\dfrac{P(B \vert A_i) \cdot P(A_i)}{P(B)} \\
  & = & \dfrac{\Big(\displaystyle\sum P(B \vert A_1, \ldots, A_N) \cdot P(A_1)\cdot \ldots \cdot P(A_{i-1}) \cdot P(A_{i+1}) \cdot \ldots \cdot P(A_N) \Big) \cdot P(A_i)}{P(B)}
\end{eqnarray*}

\end{document}
